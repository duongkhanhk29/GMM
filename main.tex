\documentclass[color=usenames,dvipsnames]{beamer}

\mode<presentation> {

\usetheme{Madrid}
\usecolortheme{lily}
\useoutertheme{infolines}

}


\usepackage{booktabs} 
\usepackage{tikz}

\usepackage{hyperref}
\hypersetup{
    colorlinks=true,
    linkcolor=blue,
    filecolor=magenta,      
    urlcolor=cyan,
}

% Thin fonts
\usepackage{cmbright}
\usepackage[T1]{fontenc}

\definecolor{dark_grey}{gray}{0.5}
\setbeamercolor{normal text}{fg=dark_grey,bg=white}
\setbeamertemplate{navigation symbols}{}

\setbeamercolor*{palette primary}{fg=gray!100,bg=gray!10}
\setbeamercolor*{palette quaternary}{fg=gray!100,bg=gray!10}
\setbeamercolor*{palette secondary}{fg=gray!100,bg=gray!20}
\setbeamercolor*{palette tertiary}{fg=gray!100,bg=gray!10}
\setbeamercolor*{navigation symbols}{fg=white,bg=white}
\usefonttheme{default}


\setbeamertemplate{blocks}[rounded][shadow=false]
\setbeamercolor{block title}{bg=gray!10}
\setbeamercolor{block body}{fg=gray,bg=gray!10}
%\setbeamercolor{frametitle}{fg=}

\setbeamertemplate{frametitle}[default][center]

\setbeamertemplate{itemize items}[default]
\setbeamertemplate{enumerate items}[default]

\newcommand{\F}{\mathbb{F}}


\title[Generalized method of moments]{Generalized method of moments}
\author{Khanh Duong}
\institute{Personal repository}
\date{\today}
\begin{document}


\begin{frame}
  \titlepage
\end{frame}

% Uncomment these lines for an automatically generated outline.
%\begin{frame}{Outline}
%  \tableofcontents
%\end{frame}


\begin{frame}{Introduction}
    The modelling equation can be rewritten as 
    \begin{equation}
    Y_{(i,t)} = \theta_0 + \theta_1 Y_{(i,t-1)} + \theta_2 X_{(i,t)} + \mu_i + \epsilon_{(i,t)}
    \label{eq:1}
    \end{equation}
    where \(i = [1, N]\) represents many \textbf{cross-sectional units} and \(t = [1, T]\) represents few \textbf{time periods}, with \textbf{country-specific effects} \(\mu_i\) and the idiosyncratic error term \(\epsilon_{(i,t)}\). The regressors \(X_{(i,t)}\) can have different properties with respect to their correlation with the error term \(\epsilon_{(i,t)}\).

    \vskip 0.5cm
    
    \begin{block}{Classification of regressors}
    The choice of whether to treat the regressors as \textbf{strictly exogenous}, \textbf{predetermined}, or \textbf{exogenous} depends on the nature of the data and the research question being addressed.
    \end{block}

\end{frame}

\begin{frame}{Strictly exogenous regressors}
    \begin{itemize}
    
        \item If the regressors are strictly exogenous, it means that the variables \(X\) are uncorrelated with \textbf{past}, \textbf{present}, and \textbf{future} values of the error term.
        
        \item In other words, \(E[X_{it}\epsilon_{is}] = 0\) for all \(t\) and \(s\).
        
        \item This implies that the values of \(X\) do not depend on the values of the error term at any point in time.
    
    \end{itemize}

\end{frame}

\begin{frame}{Predetermined regressors}
    \begin{itemize}
    
        \item If the regressors are predetermined, it means that the variables \(X\) are correlated with \textbf{past} values of the error term but \textbf{not present nor future} values.
        
        \item In other words, \(E[X_{it} \epsilon_{is}] \neq 0\) for \(s < t\), but \(E[X_{it} \epsilon_{is}] = 0\) for \(s \ge t\).
    
    \end{itemize}

\end{frame}

\begin{frame}{Endogenous regressors}
    \begin{itemize}
    
        \item If the regressors are endogenous, it means that the variables \(X\) are correlated with \textbf{past} and \textbf{present} values of the error term but \textbf{not future} values.
        
        \item In other words, \(E[X_{it} \epsilon_{is}] \neq 0\) for \(s < t\), but \(E[X_{it} \epsilon_{is}] = 0\) for \(s \ge t\).
    
    \end{itemize}

\end{frame}

\begin{frame}{Country-specific effects}
    To account for the \textbf{country-specific effects} \(\mu_i\), we employ the first difference transformation to Equation \ref{eq:1} into
    \[
    \Delta Y_{(i,t)} = \theta_1 \Delta Y_{(i,t-1)} + \theta_2 \Delta X_{(i,t)} + \Delta \epsilon_{(i,t)}
    \]
\end{frame}

\begin{frame}{The method of moments (1/2)}

    \begin{itemize}
    
        \item Afterwards, we generate \textbf{instruments} for the lagged dependent variable by using the second and third lags of $Y$, either as differences or lagged levels.
        
        \item Utilizing the \textbf{Arellano–Bond approach\footnote{Arellano, M. and Bond, S., 1991. Some tests of specification for panel data: Monte Carlo evidence and an application to employment equations. \textit{The review of economic studies}, 58(2), pp.277-297.}}, we construct instrumental variables $Z_i$ by imposing these moment conditions on the first difference model: $E[Y_{i,t-s} \Delta \epsilon_{it}] = 0$ for $s = 2, t$.
        
        \item When the individual effect term exhibits high variance across individual observations, the Arellano-Bond estimator may yield \textbf{poor performance} in finite samples.
    
        \item This occurs because the lagged dependent variables become \textbf{weak instruments} under such circumstances.
    
    \end{itemize}
\end{frame}

\begin{frame}{The method of moments (2/2)}

    \begin{itemize}
        \item \textbf{Blundell and Bond\footnote{Blundell, R. and Bond, S., 1998. Initial conditions and moment restrictions in dynamic panel data models. \textit{Journal of econometrics}, 87(1), pp.115-143.}} derived a condition that allows for the utilization of an additional set of moment conditions for the level model: $E[\Delta Y_{i,t-1} \Delta \epsilon_{it}] = 0$ for $t = 2, T$.
        
        \item Similarly, depending on the nature of the variables $X$, other instruments for those variables are incorporated as well.
        
        \item Overall, we assemble the stacked moment conditions as: $E[Z_i'\Delta \epsilon_i] = 0$, where the instrumental variables $Z_i$ are constructed from values of $X$ and $Y$.

        \item Hence, we refer to this approach as \textbf{the method of moments}.
        
    \end{itemize}
    
\end{frame}

\begin{frame}{The GMM estimator (1/2) }
    \begin{itemize}
        \item The moment conditions form a system of equations with unknown coefficients $\theta$: $E[Z_i'\Delta \epsilon_i] = E[m_i (\theta)] = 0$.
        
        \item It is evident that the vector of moment conditions ($m_i$) is larger than the vector of coefficients $\theta$, meaning that $E[m_i (\theta)]$ cannot simultaneously satisfy the condition of being equal to zero.

        \item Therefore, we aim to minimize the squared distance between the sample moment conditions and zero, which can be represented as $\| \hat{m}_i(\theta) \|_W^2 = \hat{m}_i(\theta)^T W \hat{m}_i(\theta) = f(\theta)$, where $W$ is \textbf{the weight matrix of moments} and $\hat{m}^T$ denotes transposition.

        \item By using a generalized metric for moment conditions $f(\theta)$, this method is referred to as the \textbf{generalized method of moments}.

    \end{itemize}
\end{frame}

\begin{frame}{The GMM estimator (2/2)}
    \begin{itemize}
        \item The minimal value of $f(\theta)$ occurs when its derivative with respect to $\theta$ is equal to zero, i.e., $\frac{df}{d\theta} = 0$. Obviously, the GMM estimator depends on the choice of the weight matrix $W$.

        \item A commonly used proposal for the weight matrix is $\hat{W} = \left( \frac{1}{N} Z'HZ \right)^{-1}$, where $Z$ is the instrument matrix. Under the Blundell and Bond approach for the system GMM estimator, $H$ is equal to the identity matrix ($I$).

        \item The estimation generated by this method is called the \textbf{one-step system} GMM estimator, and its weighting matrix is denoted as $\hat{W}_1$.

        \item The \textbf{two-step estimator} utilizes $\hat{W}_2 = \left( \frac{1}{N} Z'\hat{s}_1 \hat{s}_1' Z \right)^{-1}$, where $\hat{s}_1$ represents the residuals obtained from the one-step estimation.

    \end{itemize}
    
\end{frame}

\begin{frame}{Robustness (1/4)}
    \begin{itemize}
        \item The absence of higher-order serial correlation in $\Delta \epsilon_{it}$ is crucial for the validity of using $Y_{i,t-2}$, $Y_{i,t-3}$, and other variables as instruments in the GMM framework.

        \item Similarly, it is important for the instruments of predetermined and endogenous variables.

        \item To test for this, the \textbf{Arellano-Bond serial-correlation test} should be conducted. The test statistic, following an asymptotic $N(0,1)$ distribution, examines the null hypothesis $H_0: \text{Corr}(\Delta \epsilon_{it}, \Delta \epsilon_{it-j}) = 0$ for $j > 0$.

        \item If the null hypothesis is rejected for $j = 1$ but not rejected for $j > 1$, it suggests that the model passes this specification test.

    \end{itemize}
\end{frame}

\begin{frame}{Robustness (2/4)}
    \begin{itemize}
        \item Additionally, in overidentified models (e.g. GMM), where the number of moment conditions $L$ exceeds the number of unknown coefficients $K$, it is important to test the validity of the $L - K$ overidentifying restrictions. .

        \item These tests assume that at least $K$ instruments are valid.

        \item The \textbf{Sargan overidentification test} is commonly used for this purpose.

        \item The test statistic follows an asymptotic $\chi^2$ distribution with $df$ degrees of freedom, where $df$ is equal to $L - K$.

        \item The Sargan overidentification test statistic $J(\hat{\theta}, W)$ is calculated as: $\left( \frac{1}{\sqrt{N}} \sum_{i=1}^{N} m_i (\hat{\theta}) \right)' W \left( \frac{1}{\sqrt{N}} \sum_{i=1}^{N} m_i (\hat{\theta}) \right)$, where $N$ represents the number of observations or individuals in the sample.

    \end{itemize}
\end{frame}

\begin{frame}{Robustness (3/4)}
    \begin{itemize}
        \item In the presence of \textbf{heteroskedasticity}, panel-robust standard errors can be computed using system GMM estimation. 

        \item In this case, the one-step GMM estimator remains consistent under heteroskedasticity but is no longer efficient.

        \item The two-step standard errors are biased in finite samples, so the Windmeijer finite-sample correction should be applied.

        \item The corrected standard errors are still biased but less severely.

        \item However, when using panel-robust standard errors, the Sargan overidentification test cannot be computed because the asymptotic distribution is unknown.
        
    \end{itemize}
\end{frame}

\begin{frame}{Robustness (4/4)}
    \begin{itemize}
        \item It should be noted that \textbf{multicollinearity} is not a problem when using instrumental analysis, which isolates the effect of explanatory variables from group effects and other variable effects. 

        \item In instrumental analysis, the focus is on the strength of the instruments rather than the correlation among independent variables.

        \item Multicollinearity among exogenous independent variables (i.e., variables not affected by endogeneity) generally does not affect the validity of the instruments or the identification strategy used in instrumental analysis.

        \item However, perfect multicollinearity among the instruments themselves can weaken their ability to address endogeneity, and system GMM will \textbf{drop variables with perfect multicollinearity}.
        
    \end{itemize}
\end{frame}

\end{document} 